\documentclass[11pt, a4paper]{article}
\usepackage[slovene]{babel}
\usepackage[utf8]{inputenc}
\usepackage[T1]{fontenc}
\usepackage{lmodern}
\usepackage{amsmath}
\usepackage{amsfonts}
\usepackage{graphicx}
\usepackage[left=2cm, right=2cm, top=2cm, bottom=2cm]{geometry}
\usepackage{float}

\title{Modelska analiza II\\\footnotesize Naloga 1 -- Navadne diferencialne enacbe: zacetni problem}
\author{Marcel Čampa, 27182042}
\date{\today}

\begin{document}
    \maketitle

    \section*{Naloga 1}
    Pri prvi nalogi smo spremljali gibanje planeta okrog sonca. Ce zapisemo \textit{Newtonov
    gravitacijski zakon}, dobimo enacbo
    $$ m \ddot{\mathbf{r}} = -G \frac{mM}{r^3} \mathbf{r},$$
    kjer je $G$ gravitacijska konstanta, $m$ masa planeta, $M$ masa sonca, $\mathbf{r}$ pa vektor razdalje med planetom in soncem.
    Ce zapisemo se $\mathbf{r} = (x, y)^T$ in $\dot{\mathbf{r}} = (u, v)^T$, lahko zgornjo enacbo prepisemo v
    sistem
    \begin{align*}
        \dot{x} &= u\\
        \dot{y} &= v\\
        \dot{u} &= -G \frac{Mx}{(x^2 + y^2)^{1.5}}\\
        \dot{v} &= -G \frac{My}{(x^2 + y^2)^{1.5}}.
    \end{align*}
    Izberemo si zacetne pogoje $$x(0) = a,~~~ y(0) = 0, ~~~u(0) = 0,~~~v(0) = a\omega v_0,$$
    torej da planet zacne v tocki $\mathbf{r} = (a, 0)^T$, z gibanjem pa zacne navpicno navzgor ob predpostavki, da je
    $v(0) > 0$. Zaradi poenostavitve sem si izbral $G = M = a = \omega = 1$ in opazoval trajektorije
    pri razlicnih $v_0$, kar je predstavljeno na sliki \ref{fig:task1_trajectory}. \textit{Opomba: Celo nalogo sem reseval
    s \texttt{scipy.integrate.ode} z metodo \texttt{dopri5}, kar je pravzaprav Runge-Kutta.}

    \begin{figure}[H]
        \centering
        \includegraphics[width=0.45\textwidth]{../images/to_use/task1_trajectory.pdf}
        \caption{Trajektorije za razlicne $v_0$. Izkaze se, da je trajektorija za $v_0 \in (0, 1) \cup (1, \sqrt{2})$ elipsa,
        za $v_0 = 1$ kroznica, za $v_0 = \sqrt{2}$ hiperbola in za $v_0 > \sqrt{2}$ parabola. Ker je kroznica le posebna oblika elipse, lahko recemo, da je tir
        elipsa do mejne vrednosti $v_0 = \sqrt{2}$, kjer se elipsa odpre in postane hiperbola, za vecje $v_0$ pa se odpre v parabolo.}
        \label{fig:task1_trajectory}
    \end{figure}

    Nato sem se lotil malo analize. V odvisnosti od koraka $\text{d}t$ sem si pogledal trajektorije. Pricakovano se
    manjsi $v_0$ zahtevajo manjsi $\text{d}t$, saj se za majhne $v_0$ tir pribliza blizu tocke $(0,0)^T$, kar pomeni, da problem postane nestabilen.
    To je lepo prikazano na sliki \ref{fig:task1_errs_elliptic}.

    \begin{figure}[H]
        \centering
        \includegraphics[width=0.6\textwidth]{../images/to_use/err_elliptic_dt_v0.pdf}
        \caption{Opazimo, da pri manjsih $v_0$ napaka trajektorije ostane tudi do manjsih $\text{d}t$, dokler potem koncno ,,ne izgine''.}
        \label{fig:task1_errs_elliptic}
    \end{figure}

    Ogledal sem si se \textbf{napako radija v odvisnosti od dolzine koraka} $\text{d}t$. Seveda je pricakovano ta napaka zanemarljivo majhna, vendar vseeno lepo narasca, potem pa se pri $\text{d}t = 2\pi$ pricakovano
    zmanjsa, vendar ne cisto do $0$. To lepo prikazuje slika \ref{fig:task1_err_points} (levo). Potem sem pa namesto izracunanih tock vzel tocke na polovici
    med izracunanimi, kar nam da boljsi obcutek glede napake kot prejsnji izracun. Spet napaka pricakovano narasca, tokrat pa se obrne ze pri $\text{d}t = \pi$, kar je tudi ocitno,
    saj se sredisce daljice med izracunanimi tockami do $\text{d}t = \pi$ priblizuje srediscu $(0,0)^T$, potem pa spet zacne zmanjsevati. To je prikazano na sliki \ref{fig:task1_err_points} (desno).

    \begin{figure}[H]
        \centering
        \includegraphics[width=0.4\textwidth]{../images/to_use/error_circular_wr_dt_calculated_7.pdf}
        \includegraphics[width=0.4\textwidth]{../images/to_use/error_circular_wr_dt_7.pdf}
        \caption{Napaka med kroznico $\mathcal{K}((0,0)^T, 1)$ in izracunanimi tockami krivulje (levo) in napaka med kroznico $\mathcal{K}((0,0)^T,1)$ in sredisci daljic med izracunanimi tockami
        (desno). Na desni sliki je tudi levi graf.}
        \label{fig:task1_err_points}
    \end{figure}

    Za konec sem analiziral se \textbf{napako radija, energije in vrtilne kolicine v odisnosti od iteracije}.
    Vse tri grafe je bilo potrebno narisati na logaritemski skali, drugace ne bi videli nic pametnega. Grafi so prikazani na sliki \ref{fig:task1_stabilities}, napake smo izracunali s formulami
    \begin{itemize}
        \item napaka razdalje: $r_{\text{err}} = |1 - \sqrt{x^2 + y^2}|$,
        \item napaka energije: $E_{\text{err}} = |E - E_0| = \left| \left(\frac{1}{2} (x^2+y^2) - 1 \right) + \frac{1}{2} \right|$ in
        \item napaka vrtilne kolicine: $L_{\text{err}} = |L_z - L_{z_0}| = |(xv - yu) - 1|$.
    \end{itemize}

    \begin{figure}[H]
        \centering
        \includegraphics[width=0.5\textwidth]{../images/to_use/task1_stability_r.pdf}\\
        \includegraphics[width=0.5\textwidth]{../images/to_use/task1_stability_E.pdf}\\
        \includegraphics[width=0.5\textwidth]{../images/to_use/task1_stability_L.pdf}
        \caption{Odvisnost napake razdalje (zgoraj), energije (v sredini) in vrtilne kolicine (spodaj) v odvisnosti od casa
        potovanja -- iteracije.}
        \label{fig:task1_stabilities}
    \end{figure}

    Za konec pa sem se malo poigral se z razlicnimi $\text{d}t$ in dobil precej zanimive slike, nekaj od njih je na sliki
    \ref{fig:task1_interesting}. Seveda te slike nimajo nobenega fizikalnega smisla, sem jih pa vseeno vkljucil zaradi zanimivosti :).

    \begin{figure}[H]
        \centering
        \includegraphics[width=0.4\textwidth]{../images/to_use/trajectory_dt=pipolovic.pdf}
        \includegraphics[width=0.4\textwidth]{../images/to_use/trajectory_dt=pi.pdf}\\
        \includegraphics[width=0.85\textwidth]{../images/to_use/trajectory_dt=10.pdf}
        \caption{Pri $\text{d}t = \pi/2$ dobimo kvadrat (zgoraj levo) in za $\text{d}t = \pi$ daljico (zgoraj desno).
        Pri obeh pa vidimo napako, ki se zgodi, saj kvadrat ni cisti kvadrat, predvsem pa vidimo pri daljici, kako se pocasi zamika po krogu.
        Za $\text{d}t = 10$ pa dobimo precej lep vzorec (spodaj).}
        \label{fig:task1_interesting}
    \end{figure}

    \section*{Naloga 2}

    Pri tej nalogi sem zasledoval gibanje planeta okrog sonca, pri cemer vmes mimo pride mimobezna zvezda, ki se giblje po
    premici $$\mathbf{r}_z(t) = \begin{bmatrix}-10 + 2v_0 t\\ 1.5 \end{bmatrix}.$$ Iz Newtonovega zakona dobimo
    $$ m\ddot{\mathbf{r}} = -G \frac{mM}{r^3} \mathbf{r} - G\frac{mM}{|r - r_z|^3} (\mathbf{r} - \mathbf{r}_z).$$
    Zopet dobljeno prepisemo v sistem stirih enacb in dobimo
    \begin{align*}
        \dot{x} &= u\\
        \dot{y} &= v\\
        \dot{u} &= - G \frac{Mx}{(x^2 + y^2)^1.5} - G \frac{M(x-x_z)}{\left( (x-x_z)^2 + (y-y_z)^2 \right)^1.5}\\
        \dot{v} &= - G \frac{My}{(x^2 + y^2)^1.5} - G \frac{M(y-y_z)}{\left( (x-x_z)^2 + (y-y_z)^2 \right)^1.5}.
    \end{align*}
    Zaradi poenostavitve si spet izberemo $G = M = a = \omega = 1$. Zacetni pogoji se tokrat malo spremenijo, in sicer jih bomo
    zapisali v odvisnosti od zacetne faze planeta $\varphi$:
    $$ x(0) = -\sin \varphi,~~~ y(0) = \cos \varphi,~~~ u(0) = \pm v_0 \cos \varphi, ~~~ v(0) = \pm v_0 \sin\varphi,$$
    kjer bomo vzeli $v_0 = 1$, $\pm$ v $u(0)$ in $v(0)$ pa nam definirata, v katero smer se bo gibal planet (pri $+$ se giblje v negativni smeri, torej
    smeri urinega kazalca). Slika \ref{fig:task2_negative} prikazuje gibanje v smeri urinega kazalca, slika \ref{fig:task2_positive} gibanje v nasprotni smeri
    urinega kazalca, slika \ref{fig:task2_follows} pa prikazuje primere, pri katerih se planet ujame v orbito mimobezne zvezde.
    
    \begin{figure}[H]
        \centering
        \includegraphics[width=0.4\textwidth]{../images/task2_negative/task2_trajectory_phi=0.pdf}
        \includegraphics[width=0.4\textwidth]{../images/task2_negative/task2_trajectory_phi=157079632679.pdf}\\
        \includegraphics[width=0.4\textwidth]{../images/task2_negative/task2_trajectory_phi=314159265359.pdf}
        \includegraphics[width=0.4\textwidth]{../images/task2_negative/task2_trajectory_phi=471238898038.pdf}
        \caption{Gibanje v smeri urinega kazalca za razlicne vrednosti zacetne faze $\varphi$. Opazimo, da v nekaj primerih planet pobegne od sonca, kar pa se zgodi zato,
        ker je pri tem gibanju planet dlje casa blizu mimobezne zvezde, torej le-ta dlje casa deluje nanj.}
        \label{fig:task2_negative}
    \end{figure}

    \begin{figure}[H]
        \centering
        \includegraphics[width=0.4\textwidth]{../images/task2_positive/task2_trajectory_phi=0.pdf}
        \includegraphics[width=0.4\textwidth]{../images/task2_positive/task2_trajectory_phi=157079632679.pdf}\\
        \includegraphics[width=0.4\textwidth]{../images/task2_positive/task2_trajectory_phi=314159265359.pdf}
        \includegraphics[width=0.4\textwidth]{../images/task2_positive/task2_trajectory_phi=471238898038.pdf}
        \caption{Gibanje v smeri urinega kazalca za razlicne vrednosti zacetne faze $\varphi$. Opazimo, da v nekaj primerih planet pobegne od sonca, kar pa se zgodi zato,
        ker je pri tem gibanju planet dlje casa blizu mimobezne zvezde, torej le-ta dlje casa deluje nanj.}
        \label{fig:task2_positive}
    \end{figure}

    \begin{figure}[H]
        \centering
        \includegraphics[width=\textwidth]{../images/to_use/task2_trajectory_phi=392699081699.pdf}\\
        \includegraphics[width=\textwidth]{../images/to_use/task2_trajectory_phi=13.pdf}\\
        \includegraphics[width=\textwidth]{../images/to_use/task2_trajectory_phi=417831822927.pdf}
        \caption{Pri dolocenih vrednostih $\varphi$ se planet vtiri v orbito mimobezne zvezde. Ocenimo, da to velja za $\varphi \in (1.2\pi, 1.5\pi)$.}
        \label{fig:task2_follows}
    \end{figure}
\end{document}